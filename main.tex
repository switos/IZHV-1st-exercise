%%%%%%%%%%%%%%%%%%%%%%%%%%%%%%%%%%%%%%%%%%%%%%%%%%%%%%%%%%%%%%%%%%%%%%%%%%%%%%%%
% Author : [Name] [Surname], Tomas Polasek (template)
% Description : First exercise in the Introduction to Game Development course.
%   It deals with an analysis of a selected title from the point of its genre, 
%   style, and mechanics.
%%%%%%%%%%%%%%%%%%%%%%%%%%%%%%%%%%%%%%%%%%%%%%%%%%%%%%%%%%%%%%%%%%%%%%%%%%%%%%%%

\documentclass[a4paper,10pt,english]{article}

\usepackage[left=2.50cm,right=2.50cm,top=1.50cm,bottom=2.50cm]{geometry}
\usepackage[utf8]{inputenc}
\usepackage{hyperref}
\hypersetup{colorlinks=true, urlcolor=blue}

\newcommand{\ph}[1]{\textit{[#1]}}

\title{%
Analysis of Mechanics%
}
\author{%
Sviatoslav Shishnev xshish02%
}
\date{}

\begin{document}

\maketitle
\thispagestyle{empty}

{%
\large

\begin{itemize}

\item[] \textbf{Title:} Coromon

\item[] \textbf{Released:} 31.03.2022

\item[] \textbf{Author:} Game studio: TRAGsoft, publisher: Freedom Games 

\item[] \textbf{Primary Genre:} Role play game

\item[] \textbf{Secondary Genre:} Puzzle

\item[] \textbf{Style:} Pixel-cartoon style 

\end{itemize}

}

\section*{\centering Analysis}

\subsection*{Content}

\begin{enumerate}
    \item How are the primary and secondary genres reflected in the gameplay? \\ \\ While playing Coromon, you engage heavily in the role-playing aspect during combat actions. Your role is that of an explorer in a world filled with mysterious creatures called Coromons, and your goal is to gather all possible data by capturing them in ball form storage. However, before you can catch a Coromon, you need to weaken it by engaging in battles—a form of tactical turn-based combat. In these battles, you can strategically choose which of your creatures will be in action and determine their fighting approach. Building a solid strategy for your team is crucial to overcoming the various challenges the game presents. The game essentially includes elements of turn-based tactical strategy.\\ \\
    However, I consider the secondary genre to be puzzle. When you're not battling wild creatures and your colleagues, that are eager to participate in battles for training and new data, you may encounter puzzles. These puzzles could be in the form of text with conditions to meet, a series of traps to avoid, or a memory game to unlock doors.
    
    \item How do the primary and secondary genre interact? Do the secondary genres support the primary genre? Do they enhance the game, or are they detrimental? \\ \\ I believe the puzzle aspect is supportive. After engaging in tactical fights repeatedly, solving an easy puzzle can inject new life into the gameplay. It's a joy to overcome challenges and receive rewards through puzzle-solving. However, there's a distinct part of the game where you encounter a boss, and puzzles that hinder your progress towards facing the boss can be quite annoying for some players (including myself).
    
    \item Does the style support the gameplay? Why was it chosen?\\\\  Perhaps as you read through this text, you observed the game's striking resemblance to legendary and ancient Pokémon games like the Emerald, Ruby, Sapphire series, or Fire Red and Leaf Green versions. Coromon essentially mirrors their gameplay, involving catching creatures, solving puzzles, and engaging in battles with trainers or explorers. The pixel art style was deliberately chosen to evoke nostalgia within the target audience, including myself. The original series utilized a pixel style due to the constraints of computer power during that era. While creating pixel graphics today may be more challenging than developing 3D models and assets. This style is strongly associated with the JRPG genre, particularly the original Pokémon series. In my opinion, the pixel art style effectively captures a friendly atmosphere, resonating with the essence of games like Coromon. I believe, the chosen style is supportive, enhancing the overall gaming experience
\end{enumerate}

\end{document}
